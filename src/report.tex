\documentclass{article}

% Language setting
\usepackage[english]{babel}

% Set page size and margins
\usepackage[letterpaper,top=2cm,bottom=2cm,left=3cm,right=3cm,marginparwidth=1.75cm]{geometry}

% Useful packages
\usepackage{amsmath}
\usepackage{graphicx}
\usepackage[colorlinks=true, allcolors=blue]{hyperref}

\title{Data Sources and Retrieval}
\author{Your Name}

\begin{document}
\maketitle

\section{Overview}
The primary objective of this Bloomberg data retrieval script is to extract historical data required for \textbf{Equity Spot-Futures Arbitrage Analysis}. The primary data source for this project is the \textbf{Bloomberg Terminal}, aligning with the methodology used by the reference paper. The data extraction process is automated using the \textbf{Python xbbg package}, allowing seamless interaction with the Bloomberg Terminal API.

To ensure reproducibility, the script supports both \textbf{live data extraction} and a fallback to \textbf{cached data}. Users can toggle data retrieval behavior using the \texttt{USING\_XBBG} environment variable in the configuration file. If \texttt{USING\_XBBG=True}, data is pulled directly from Bloomberg; otherwise, the script defaults to previously saved datasets.

\section{Data Types and Key Tickers}
The script retrieves three key types of financial data:

\subsection{Spot Prices and Dividend Yields}
\begin{itemize}
  \item Required for computing spot-futures arbitrage.
  \item Extracted for \textbf{S\&P 500 (SPX Index)}, \textbf{Nasdaq 100 (NDX Index)}, and \textbf{Dow Jones Industrial Average (INDU Index)}.
  \item Bloomberg Fields:
  \begin{itemize}
    \item \texttt{PX\_LAST} -- Last traded price of the index.
    \item \texttt{IDX\_EST\_DVD\_YLD} -- Estimated annual dividend yield.
    \item \texttt{INDX\_GROSS\_DAILY\_DIV} -- Daily gross dividend.
  \end{itemize}
\end{itemize}

\subsection{Equity Index Futures Contracts}
\begin{itemize}
  \item Used to construct \textbf{implied forward rates} for arbitrage calculations.
  \item The script pulls \textbf{multiple futures contracts} per index, ordered by contract expiry:
  \begin{itemize}
    \item \textbf{Nearest (Front Month)}: ES1 Index, NQ1 Index, DM1 Index
    \item \textbf{First Deferred}: ES2 Index, NQ2 Index, DM2 Index
    \item \textbf{Second Deferred}: ES3 Index, NQ3 Index, DM3 Index
    \item \textbf{Third Deferred}: ES4 Index, NQ4 Index, DM4 Index
  \end{itemize}
  \item Bloomberg Fields:
  \begin{itemize}
    \item \texttt{PX\_LAST} -- Last traded futures price.
    \item \texttt{PX\_VOLUME} -- Trading volume.
    \item \texttt{OPEN\_INT} -- Open interest.
    \item \texttt{CURRENT\_CONTRACT\_MONTH\_YR} -- Expiry date of the futures contract.
  \end{itemize}
\end{itemize}

\subsection{Overnight Indexed Swap (OIS) Rates}
\begin{itemize}
  \item Used as a benchmark for the risk-free rate in arbitrage calculations.
  \item The script retrieves various OIS maturities, but \textbf{OIS\_3M (3-month OIS rate) is the primary reference}, as the paper states that interpolation across OIS rates is unnecessary.
  \item Bloomberg Fields:
  \begin{itemize}
    \item \texttt{PX\_LAST} -- Last recorded interest rate.
  \end{itemize}
\end{itemize}

\section{Data Extraction Process}
The script is structured into three key functions, each responsible for fetching one category of data:

\begin{enumerate}
  \item \textbf{Spot Prices and Dividend Data Retrieval}
  \begin{quote}
    \texttt{pull\_spot\_div\_data(["SPX Index"], START\_DATE, END\_DATE)}
  \end{quote}
  - Fetches index price and dividend-related fields.\\
  - Returns a multi-indexed pandas DataFrame with a timestamped index.

  \item \textbf{Futures Contract Data Retrieval}
  \begin{quote}
    \texttt{pull\_futures\_data(["ES1 Index", "ES2 Index"], START\_DATE, END\_DATE)}
  \end{quote}
  - Retrieves price, volume, and open interest for futures contracts.\\
  - Data is stored with a date-time index.

  \item \textbf{OIS Rate Retrieval}
  \begin{quote}
    \texttt{pull\_ois\_rates(["USSOC CMPN Curncy"], START\_DATE, END\_DATE)}
  \end{quote}
  - Pulls OIS rates for different maturities.\\
  - Primarily focuses on 3-month OIS.
\end{enumerate}

Once extracted, all datasets are merged into a \textbf{single time-series DataFrame} with a uniform date range defined by \texttt{START\_DATE} and \texttt{END\_DATE}.

\section{Data Storage and Logging}
\begin{itemize}
  \item Extracted data is \textbf{saved as a Parquet file} for efficient storage and retrieval.
  \item The script logs all activities using Python's \texttt{logging} module:
  \begin{itemize}
    \item Logs are stored in \texttt{\_output/temp/bloomberg\_data\_extraction.log}.
    \item Errors are captured and printed to the console for debugging.
  \end{itemize}
\end{itemize}

\section{Challenges and Considerations}
During development, several issues were encountered with the \texttt{xbbg} package:
\begin{itemize}
  \item \textbf{Import Errors (blpapi not found)}
  \begin{itemize}
    \item Fixed by ensuring installation via:
    \begin{quote}
      \texttt{conda install -c conda-forge blpapi}
    \end{quote}
    \item Added to \texttt{requirements.txt} for reproducibility:
    \begin{quote}
      \texttt{blpapi @ https://blpapi.bloomberg.com/repository/releases/python/simple/}
    \end{quote}
  \end{itemize}
  \item \textbf{Handling Asynchronous Market Closing Times}
  \begin{itemize}
    \item The script avoids spot/futures price mismatch by using \textbf{only futures-based implied forward rates} rather than direct spot-futures parity calculations.
  \end{itemize}
\end{itemize}

\section{Conclusion}
This script ensures that all \textbf{necessary raw financial data} is collected in an automated, reproducible manner. The retrieved data serves as the foundation for computing \textbf{equity spot-futures arbitrage spreads}, which will be further analyzed in subsequent scripts.

\end{document}